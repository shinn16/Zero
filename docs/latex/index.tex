You can find all code documentation by going to the classes link above and looking at the classes. All major methods are documented. Below are some explanations on assumptions and on why things were done the way they were done.

\subsection*{Assumptions}

The pipeline follows the in-\/class examples we did. To increase the speed of the pipeline, I assumed that all register values are accessible through a register file as they are being written allowing a dependent instruction to execute one clock cycle sooner.

\subsection*{Logic}

\paragraph*{The A\+LU}

There is not a class for the A\+LU in my project. I started out with one, but then decided it was not necessary. The A\+LU is responsible for the execution phase in the pipeline, which I wrote a method for. Writing an A\+LU class would essentially be a wrapper for this method. Since it served no other purpose, I just left the method as is.

\paragraph*{The Pipeline}

Each \mbox{\hyperlink{class_pipeline_stage}{Pipeline\+Stage}} is an object wrapper. It contains a string that is the instruction to execute ( though this is only for the log print out), a \mbox{\hyperlink{class_data_bus}{Data\+Bus}} which is an object wrapper for passing data between pipeline stages, and a boolean lock. The pipeline implementation is done using 5 methods in the \mbox{\hyperlink{class_c_p_u}{C\+PU}} class. The \mbox{\hyperlink{class_data_bus}{Data\+Bus}} object for each instruction is passed between each stage until its execution is finished. The \mbox{\hyperlink{class_data_bus}{Data\+Bus}} object allows for easy handling of instruction arguments instruction decoding.

The pipeline is executing in the following order\+: write back, memory access, execution, instruction decode, and instruction fetch. In reality all of these stages execute simultaneously, but since I did not thread them I elected this order because it worked best for passing the \mbox{\hyperlink{class_data_bus}{Data\+Bus}} objects around.

\paragraph*{\mbox{\hyperlink{class_memory}{Memory}}}

My \mbox{\hyperlink{class_memory}{Memory}} array is partitioned into two sections\+: \begin{DoxyVerb}1. Program Instructions
2. Program Data
\end{DoxyVerb}


This is done to prevent the program from over writing its own instructions. The \mbox{\hyperlink{class_memory}{Memory}} is one string array of 256 length, but it is made into its own class so that I can easily handle the partitioning described above and easily manage inserting data and instructions.

\subsection*{Log File}

The log file produces a table with the pipeline stages and what instruction they are executing. Each row in the table is a clock cycle. Whenever a stall occurs, it will show as a stall in the column of the pipeline stage that it occurred at. Null values will appear when a stage is completely empty. 